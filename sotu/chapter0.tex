\chapter*{内容梗概}
\addcontentsline{toc}{chapter}{内容梗概}

本論文は,筆者が愛媛大学大学院理工学研究科電子情報工学専攻電気電子工学コースに在学中に
行った, 太陽光発電データの時刻補正手法とElasticsearchノードのクラスタ化手法の提案についてまとめたものであり, 以下の5章から構成されている.\\

\begin{quote}
      \begin{description}

            \item[第\ref{chap:first}章 緒論]\ \\
            本研究を行うに至った経緯及び, 本研究の目的について述べている.
            \vspace{3.0mm}
            
            \item[第\ref{chap:second}章 太陽光発電データの時刻補正手法]\ \\
            ここでは, 太陽光発電データの計測時刻の補正手法を提案する. また, pvlibと呼ばれる太陽光発電シュミレーターライブラリを導入して, 計測データを前処理することによる補正誤差の改善手法についても提案する.
            \vspace{3.0mm}
            
            \item[第\ref{chap:third}章 単一Elasticsearchノードをクラスタ化する前に行うデータ移行]\
            ここでは, 単一Elasticsearchノードをクラスタ化する前に行うデータ移行について述べており, CO\textsubscript{2}データの移行とLEAFの運行日誌のデータ移行プロセスを詳述し, 移行後のデータ可視化による確認を行っている.
            \vspace{3.0mm}
            
            \item[第\ref{chap:fourth}章 仮想環境を使用したクラスタリング動作の検証]\ \\
            ここでは, 仮想環境を使用したクラスタリング動作の検証について述べており, バージョンの異なるElasticsearchノードを用いたクラスタ構築の可否や, 既存Elasticsearchノードの異なるクラスタへの参加の可否の検証結果について述べる.
            \vspace{3.0mm}
            
            \item[第\ref{chap:fifth}章 結論と今後の課題]\ \\
            本研究によって明らかになった事項や今後の研究課題についてまとめている.
      \end{description}
\end{quote}
