\chapter*{内容梗概}
\addcontentsline{toc}{chapter}{内容梗概}

本論文は,筆者が愛媛大学大学院理工学研究科電子情報工学専攻電気電子工学コースに在学中に
行った, 太陽光発電データの時刻補正手法とElasticsearchのデータ移行と冗長化についてまとめたものであり, 以下の5章から構成されている。\\

\begin{quote}
      \begin{description}

            \item[第\ref{chap:first}章 緒論]\ \\
            本研究を行うに至った経緯及び, 本研究の目的について述べている。
            \vspace{3.0mm}
            
            \item[第\ref{chap:second}章 太陽光発電データの時刻補正手法]\ \\
            ここでは, 太陽光発電データの時間的ずれを特定するための相互相関を用いた時刻補正手法について述べており, 大気外日射量の計算と実測データの比較を通じて, 時刻補正の効果とその必要性を示す. 
            \vspace{3.0mm}
            
            \item[第\ref{chap:third}章 学内ゾーンのElasticsearchクラスタへのデータ移行]\
            ここでは, Elasticsearchを使用した学内ゾーンのデータ移行手順と重複データの削除方法について述べており, CO2濃度データの移行とLEAFの運行日誌のデータ移行プロセスを詳述し, 移行後のデータ可視化による成功の確認を行っている.
            \vspace{3.0mm}
            
            \item[第\ref{chap:fourth}章 サーバーゾーンでのクラスタ構築]\ \\
            ここでは, 異なるバージョンのElasticsearchノードを用いたクラスタリングの可能性と, 既存ノードの新しいクラスタへの参加可能性について検証した結果について述べており, バージョンの不一致がクラスタリングに与える影響と, 既存ノードのクラスタ参加に関する制約が明らかにしている.
            \vspace{3.0mm}
            
            \item[第\ref{chap:fifth}章 結論]\ \\
            本研究によって明らかになった事項や今後の研究課題についてまとめている。
      \end{description}
\end{quote}
