\chapter{緒論}
\label{chap:first}

%%序論を書いてください. 卒論の内容を簡単にまとめてください. 
太陽光発電は, 再生可能エネルギー源として世界中で注目されており, 効率的な運用と管理には正確な計測データが不可欠である. しかしながら, 本研究で管理している太陽光発電データは, データを計測しているPCの内部時計が標準時刻とずれているため, 他地点で計測しているデータと時刻同期ができない問題があった.

本論文では, この問題に対処するため, 太陽光発電データの計測日時の補正手法を提案する. また, pvlibと呼ばれる太陽光発電シュミレーターライブラリを導入して, 計測データを前処理することによる補正誤差の改善手法についても提案する.
次に, 発電データ等を蓄積しているElasticsearchサーバをクラスタ化して故障耐性を向上する際に必要な作業の実施方法を提案する. 単一で動いているサーバのデータをクラスタ化サーバにするためのプロセスについて提案する. バージョンの異なるElasticsearchノードを用いたクラスタ構築の可否や, 既存Elasticsearchノードの異なるクラスタへの参加の可否の検証結果も述べる. 

本論文は以下の構成となっている. 第2章では, 太陽光発電データの計測日時を標準時に補正するために, 理論値との相互相関を用いる手法を提案する. 第3章では, 単一で動いているサーバのデータをクラスタ化するためのプロセスについて提案する. 第4章では, バージョンの異なるElasticsearchノードを用いたクラスタ構築の可否や, 既存Elasticsearchノードの異なるクラスタへの参加の可否の検証結果も述べる. 第5章では, 本研究の成果と課題を明らかにする.