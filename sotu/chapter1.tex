\chapter{緒論}
\label{chap:first}

%%序論を書いてください. 卒論の内容を簡単にまとめてください. 
急速に進化するデータ管理において、Elasticsearch は極めて重要な技術として登場し、データの保存、検索、管理方法に革命をもたらした. 当初、大量のデータを効率的に処理するために設計された Elasticsearch は、シンプルなシングルノードシステムから複雑なクラスタ構成へと移行し、データハンドリング技術において大きな進歩を遂げました(1). この変遷は、様々な分野で堅牢でスケーラブル、かつ冗長性のあるデータ管理システムに対する要求が高まっていることを裏付けている. 

データシステムの冗長性、特に Elasticsearch クラスタにおける冗長性は、データの信頼性と可用性を確保する上で重要なポイントとなっています. 冗長性とは、システムの重要なコンポーネントや機能を二重化することで、信頼性を高め、一点障害のリスクを低減することを指す(2). 学内やサーバーゾーンのような複雑な環境では、効果的な冗長性の実現は、特にデータ移行やシステムの拡張性という観点から、独自の課題を提起する. 研究によると、クラスタ化されたシステムにおけるデータ冗長化の取り組みの最大30\%が、初期導入段階で課題に直面しており、このような取り組みの複雑さを浮き彫りにしています(3). 

Elasticsearch のノード管理における現在のトレンドは、クラスタ化されたシステムを好む傾向が強まっていることを示しています. しかし、特に学内やサーバゾーンのような分離された環境において、これらのシステムへのデータ移行に関する包括的なドキュメントや分析にはギャップがあります. Elasticsearchの技術的な側面については多くの研究がありますが、同じ環境内でシングルノードシステムからクラスタ化されたシステムへデータを移行し、さらに別のゾーンに新しいクラスタ化されたシステムを構築する際の実際的な課題や戦略について掘り下げた研究はほとんどありません(4). 

本研究の目的は、シングルノードの Elasticsearch システムから学内ゾーン内のクラスタ化システムへデータを移行するプロセスを分析することで、このギャップを埋めることである. 同時に本研究では、この2つのゾーン間のデータ移行を行わずに、サーバゾーンに新しい冗長化されたクラスタ化システムを構築することを検討する. この二重のアプローチにより、環境をまたいだデータ移行の複雑さを排除した冗長性と信頼性に焦点を当てた、Elasticsearchデータ管理に関するユニークな視点が提供されます. 

この研究で期待される成果は、Elasticsearch環境におけるマイグレーションプロセスの包括的な理解と、別々のゾーンにおける冗長性の確立に関する洞察です. 方法論的に、本研究はケーススタディーアプローチを採用しており、Elasticsearch システムにおけるデータ移行と冗長性の実装の具体例を詳細に分析することができます. このようなアプローチは、より広範な調査や実験的研究(5)では失われがちな、これらの操作に関わる微妙なプロセスや意思決定を解明する上で特に価値がある. 

さらに、この研究はデータ管理のより広い分野に重要な意味を持つ. 得られた洞察は、様々な環境、特に高レベルのデータの信頼性とシステムの冗長性を必要とする環境における将来のElasticsearchの実装に役立つ可能性がある. さらに、これらの知見は、同様のクラスタ化環境におけるデータ移行プロセスや冗長性戦略の最適化に関する更なる研究に拍車をかける可能性があり、学術研究から産業レベルのデータ管理まで幅広いアプリケーションに恩恵をもたらす可能性がある(6). 

要約すると、本研究は Elasticsearch におけるデータ管理の重要な側面である、冗長性を目的としたクラスタシステムへのデータ移行と、異なる環境における別個の、しかし機能的には類似したシステムの確立という、特定の、しかし重要な側面に焦点を当てている点で際立っている. この焦点は、詳細なケーススタディーアプローチと組み合わされ、この分野へのユニークな貢献を提供し、実務家と研究者の両方に貴重な洞察を提供します. 
 第2章では学内ゾーンにおけるElasticsearchノードから新クラスタへのデータ移行について述べる.第3章では仮想環境を使用してサーバーゾーンのElasticsearchノードをシミュレートし、マルチコンテナDockerアプリケーションのツールであるdocker-composeを用いてクラスタ構築の実現可能性を検証したことについて述べる. 第4章ではサーバーゾーンで既存のElasticsearchノードを用いたクラスタ構築について述べる. 第5章では結論と課題を述べる. 