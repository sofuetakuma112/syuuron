\chapter{緒論}
\label{chap:first}

%%序論を書いてください. 卒論の内容を簡単にまとめてください. 
急速に進化するデータ管理において、Elasticsearch は極めて重要な技術として登場し、データの保存、検索、管理方法に革命をもたらした. 当初、大量のデータを効率的に処理するために設計された Elasticsearch は、シンプルなシングルノードシステムから複雑なクラスタ構成へと移行し、データハンドリング技術において大きな進歩を遂げた. この変遷は、様々な分野で堅牢でスケーラブル、かつ冗長性のあるデータ管理システムに対する要求が高まっていることを裏付けている. 

データシステムの冗長性、特に Elasticsearch クラスタにおける冗長性は、データの信頼性と可用性を確保する上で重要な要素となっている. 冗長性とは、システムの重要なコンポーネントや機能を二重化することで、信頼性を高め、一点障害のリスクを低減することを指す.

Elasticsearch のノード管理における現在のトレンドは、クラスタ化されたシステムを好む傾向が強まっている. しかし、Elasticsearchの技術的な側面については多くの研究があるが、同じ環境内でシングルノードシステムからクラスタ化されたシステムへデータを移行し、さらに別のネットワークゾーンに新しくクラスタ化されたシステムを構築する際の実際的な課題や戦略について掘り下げた研究はほとんどない. 

本研究の目的は、シングルノードの Elasticsearch システムから学内ゾーン内のクラスタ化システムへデータを移行するプロセスを分析することである. 同時に本研究では、この2つのゾーン間のデータ移行を行わずに、サーバゾーンに新しい冗長化されたクラスタ化システムを構築する. 

第2章では学内ゾーンにおけるElasticsearchノードから新クラスタへのデータ移行について述べる. 第3章では仮想環境を使用してサーバーゾーンのElasticsearchノードをシミュレートし、マルチコンテナDockerアプリケーションのツールであるdocker-composeを用いてクラスタ構築の実現可能性を検証したことについて述べる. 第4章ではサーバーゾーンで既存のElasticsearchノードを用いたクラスタ構築について述べる. 第5章では結論と課題を述べる. 