\chapter{緒論}
\label{chap:first}

%%序論を書いてください. 卒論の内容を簡単にまとめてください. 
太陽光発電は, 再生可能エネルギー源として世界中で注目されており, その効率的な運用と管理には正確な計測データが不可欠である. しかしながら, 実際の計測データには様々な課題が存在する. 特に, PCの内部時計のずれによる時間情報の不正確さは, データの信頼性を損なう重要な問題である.

本研究では, これらの問題に対処するため, 太陽光発電の計測データの補正手法を提案する. 具体的には, 相互相関を用いて計測データの時間的ずれを特定し, 前処理の追加やpvlibライブラリを使用した日射量の計算によって, 相互相関の計算精度を向上させる. また, 本研究室で運用しているElasticsearchのデータ移行と, サーバーゾーンにおけるクラスタ構築に向けた事前検証も行う.

本論文は以下の構成となっている. 第2章では, 太陽光発電の計測データの時間的ずれの特定に焦点を当て, 相互相関を用いた手法を提案する. 第3章では, Elasticsearchを用いたデータ管理と移行について詳述する. 第4章では, 仮想環境を使用したクラスタ構築の検証について述べ, 特にバージョンの異なるElasticsearchノード間でのクラスタリングに関する検証を行う. さらに, ElasticsearchおよびKibanaのバージョンアップとその後の動作確認についても詳細に説明する. これらの章を通じて, 太陽光発電データの精度と, Elasticsearchの最適化を目指す本研究の成果と課題を明らかにする.