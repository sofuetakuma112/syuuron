\chapter{結論と今後の課題}
\label{chap:fifth}

\section{結論}
本研究では, 太陽光発電の計測データ補正とElasticsearchのデータ移行および冗長化について詳細に検討し, 以下の主要な成果を達成した.

% 本研究では, 単一ノードのElasticsearchシステムから学内ゾーン内のクラスタ化システムへデータを移行するプロセスを分析し, さらにサーバゾーンに新しい冗長化されたクラスタ化システムを構築する検証を行った. 

% データ移行プロセスの検証では, CO2データとLEAFの運行日誌に関するデータの移行を行った. CO2データの移行では, 重複データの削除が必要であったが, SQLiteデータベースを用いた手法で対応した. LEAFの運行日誌に関するデータの移行では, 同じ名前のインデックスを移行先のElasticSearchサーバーに作成して, データをインサートすることで行った. 

% サーバーゾーンでのクラスタ構築の検証では, Docker, Docker Composeを使用した事前検証を行った. 異なるバージョンのElasticsearch(7.17.6と7.17.9)を使用したクラスタの構築を試したが, 正常に構築できないことを確認した. 

\begin{itemize}
  \item 相互相関を用いた太陽光発電計測データの時間的ずれの特定手法の提案.
  \item Elasticsearchクラスタへのデータ移行に関する具体的な手順の確立と成功による, データ管理とアクセスの効率化.
  \item サーバーゾーンでのElasticsearchクラスタ構築に向けた仮想環境を使用した事前検証を通じて, バージョンアップの重要性と手順の確立.
\end{itemize}

\section{今後の課題}
本研究の成果を踏まえ, 今後の研究の方向性として以下の課題が考えられる.

\begin{itemize}
  \item 太陽光発電計測データの補正手法のさらなる改善.
  \item 学内ゾーンとサーバーゾーンでそれぞれ稼働しているクラスタごとにkibanaが存在しており, 本研究室で管理するElasticsearchに保存されたデータを一元的に管理, 閲覧することが出来ないので, kibanaの統合による一元管理の実現.
  % \item Elasticsearchクラスタのセキュリティ対策とデータバックアップ戦略の強化.
  \item システムの継続的なモニタリングと定期的なメンテナンスの実施.
\end{itemize}
