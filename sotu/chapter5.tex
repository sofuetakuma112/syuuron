\chapter{結論と今後の課題}
\label{chap:fifth}

\section{結論}
% 本研究では, 太陽光発電の計測データ補正とElasticsearchのデータ移行およびクラスタ化について詳細に検討し, 以下の主要な成果を達成した.

% \begin{itemize}
%   \item 相互相関を用いた太陽光発電計測データの時間的ずれの特定手法の提案.
%   \item Elasticsearchクラスタへのデータ移行に関する具体的な手順の確立と成功による, データ管理とアクセスの効率化.
%   \item サーバーゾーンでのElasticsearchクラスタ構築に向けた仮想環境を使用した事前検証を通じて, バージョンアップの重要性と手順の確立.
% \end{itemize}

% \section{今後の課題}
% 本研究の成果を踏まえ, 今後の研究の方向性として以下の課題が考えられる.

% \begin{itemize}
%   \item 太陽光発電計測データの補正手法のさらなる改善.
%   \item 学内ゾーンとサーバーゾーンでそれぞれ稼働しているクラスタごとにKibanaが存在しており, 本研究室で管理するElasticsearchに保存されたデータを一元的に管理, 閲覧することが出来ないので, Kibanaの統合による一元管理の実現.
%   \item システムの継続的なモニタリングと定期的なメンテナンスの実施.
% \end{itemize}

本研究では, 太陽光発電データの時刻補正手法とElasticsearchノードのクラスタ化手法を提案した.

太陽光発電データの計測時刻を理論値(pvlib)との相互相関を用いて修正する際, 前処理を行うことにより修正誤差が74秒から27秒に改善したことを述べた. 重複があるデータをElasticsearchクラスタにデータ移行する時は, SQLiteの重複削除機能が有効であった. 本論文の場合では338GBあったデータが削除後は9.7GBに圧縮できた.  Dockerによる仮想環境を使用することで, クラスタリング動作の可否の検証が効率よく行えることを述べた. また, ElasticsearchソフトウェアのバージョンやUUIDの管理方法についての知見を得た.

今後の課題としては, pvlibに設定するパラメータをチューニングして地表日射量の計算値を実測値により近づけてから再度提案法を検証する必要がある. また, クラスタを実際に構築し, それを運用し続けるためのノウハウを蓄積する必要がある.

