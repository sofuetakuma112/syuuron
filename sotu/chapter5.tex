\chapter{結論と今後の課題}
\label{chap:fifth}
%%%結論を簡単に述べてください. 
本研究では, 単一ノードのElasticsearchシステムから学内ゾーン内のクラスタ化システムへデータを移行するプロセスを分析し, さらにサーバゾーンに新しい冗長化されたクラスタ化システムを構築する検証を行った. 

データ移行プロセスの検証では, CO2データとLEAFの運行日誌に関するデータの移行を行った. CO2データの移行では, 重複データの削除が必要であったが, SQLiteデータベースを用いた手法で対応した. LEAFの運行日誌に関するデータの移行では, 同じ名前のインデックスを移行先のElasticSearchサーバーに作成して, データをインサートすることで行った. 

サーバーゾーンでのクラスタ構築の検証では, Docker, Docker Composeを使用した事前検証を行った. 異なるバージョンのElasticsearch(7.17.6と7.17.9)を使用したクラスタの構築を試したが, 正常に構築できないことを確認した. 
%%%今後の課題について述べてください. 

学内ゾーンとサーバーゾーンでそれぞれ稼働しているクラスタごとにkibanaが存在しており, 本研究室で管理するElasticsearchに保存されたデータを一元的に管理, 閲覧することが出来ないので, kibanaの統合による一元管理が出来る状態を目指す.
