\chapter{結論と今後の課題}
\label{chap:fifth}

\section{結論}
本研究では, 太陽光発電データの同期を目的として3種類の時刻補正手法を提案した. また, Elasticsearchノードのクラスタ化手法を提案した.

太陽光発電データの計測時刻を大気外日射量との相互相関によって補正する手法を用いる場合に比べて, pvlibにより求まる地表日射量を用いれば補正誤差を78秒から68秒に改善できた. さらに実測値のうち, 日の出と日の入時刻付近のデータを削除する前処理により, 補正誤差は68秒から34秒に改善したことを述べた. 

CO\textsubscript{2}データを蓄積しているサーバには重複があるデータが多数存在していた. 4章で述べるElasticsearchクラスタにデータ移行するに先立ち行ったSQLiteによる重複削除機能が有効だったことを述べた. 本論文の場合では338GBあったデータが削除後は9.7GBに圧縮でき, 圧縮後のデータは必要以上に削除していないことを確認した.

4章では, Dockerによる仮想環境を使用することで, クラスタリング動作の可否の検証が効率よく行えたことを述べた. またその検証の結果, ElasticsearchソフトウェアのバージョンやUUIDの管理方法について得た知見を述べた.

\section{今後の課題}

今後の課題としては, pvlibに設定するパラメータをチューニングして地表日射量の計算値を実測値により近づけてから再度提案法を検証する必要がある. また, 得られた知見に基づきクラスタを実際に構築し, それを運用し続けるためのノウハウを蓄積していく必要がある.

