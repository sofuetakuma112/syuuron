
\chapter{太陽光発電の計測データの補正}
\label{chap:second}

\section{緒言}
本章では太陽光発電の計測データの補正について述べる.

% 20220523

\section{太陽光発電の計測データの問題点について}
CSVデータなどで保存された太陽光発電の環境データはオフライン環境でファイル書き込みを行っているため, PCの内部時計がずれており日時情報が誤っている場合がある.
そこで, 任意の緯度経度と日射量から日時を求めることの第一段階として, 任意の緯度経度と日時から日射量を計算するプログラムを開発した.

\section{日射量の式の導出}
任意の緯度経度, 日時における日射量$Q$は, 任意の緯度$\phi$, 経度$\lambda$の地点における任意の日時, 太陽高度$\alpha$から求めることができる.

まず, 次式により元旦からの通し日数$dn$に基いて定めた$\theta$を用いて, 当該日の太陽赤緯$\delta$, 地心太陽距離$\frac{r}{r^{*}}$, 均時差$E_q$をそれぞれ以下の式により求める.
\begin{eqnarray}
  \theta =  \frac{2\pi (dn-1)}{365}
\end{eqnarray}

\begin{eqnarray}
\begin{split}
  \delta =  0.006918-0.399912\cos \theta+0.070257\sin \theta-0.006758\cos 2\theta\\
  +0.000907\sin 2\theta-0.002697\cos 3\theta+0.001480\sin 3\theta
\end{split}
\end{eqnarray}

\begin{eqnarray}
  \frac{r}{r^{*}} =  \frac{1}{\sqrt{1.000110+0.034221\cos \theta+0.001280\sin \theta+0.000719\cos 2\theta+0.000077\sin 2\theta}}
\end{eqnarray}

\begin{eqnarray}
  E_q =  0.000075+0.001868\cos \theta-0.032077\sin \theta-0.014615\cos 2\theta-0.040849\sin 2\theta
\end{eqnarray}

日本標準時間から, 太陽の時角$h$を求める.

\begin{eqnarray}
  h = \frac{(日本標準時間-12)\pi}{12}+標準子午線からの経度差+E_q
\end{eqnarray}

$\delta$, $\phi$, $h$の値が既知となったので$\alpha$は

\begin{eqnarray}
  \alpha = \arcsin (\sin \phi\sin \delta+\cos \phi\cos \delta\cos h)
\end{eqnarray}

として求めることができる.

最後に, $Q$を

\begin{eqnarray}
  Q = 1367(\frac{r^{*}}{r})^{2}\sin \alpha
\end{eqnarray}

により求めることができる.
また, 1367\si{\watt}/\si{\metre\squared}は太陽定数である.

\section{日射量を計算するプログラムの開発}
式(1)から式(7)をもとに, 日射量を求めるプログラムを開発した.
% なお, ソースコード\ref{sc1}では式(7)の計算について, 式変形によって得られた以下の式(8), (9)を用いて計算している.

% \begin{eqnarray}
%   (\frac{r^{*}}{r})^{2} = 1.000110+0.034221\cos \theta+0.001280\sin \theta+0.000719\cos 2\theta+0.000077\sin 2\theta
% \end{eqnarray}

% \begin{eqnarray}
%   \sin \alpha = \sin \{\arcsin (\sin \phi\sin \delta+\cos \phi\cos \delta\cos h)\} = \sin \phi\sin \delta+\cos \phi\cos \delta\cos h
% \end{eqnarray}

% \begin{lstlisting}[caption=任意の緯度経度と日時から日射量を計算するプログラム, label=sc1]
% import numpy as np
% import datetime

% dt = datetime.datetime(2022, 5, 17, 17, 53)
% dt_new_year = datetime.datetime(dt.year, 1, 1)
% dt_delta = dt - dt_new_year  # 元旦からの通し日数

% dn = dt_delta.days + 1
% theta = 2 * np.pi * (dn - 1) / 365

% # 太陽赤緯(単位はラジアン)
% delta = (
%     0.006918
%     - (0.399912 * np.cos(theta))
%     + (0.070257 * np.sin(theta))
%     - (0.006758 * np.cos(2 * theta))
%     + (0.000907 * np.sin(2 * theta))
%     - (0.002697 * np.cos(3 * theta))
%     + (0.001480 * np.sin(3 * theta))
% )

% # 地心太陽距離のルートの中身
% geocentri_distance_like = (
%     1.000110
%     + 0.034221 * np.cos(theta)
%     + 0.001280 * np.sin(theta)
%     + 0.000719 * np.cos(2 * theta)
%     + 0.000077 * np.sin(2 * theta)
% )

% eq = (  # 均時差
%     0.000075
%     + 0.001868 * np.cos(theta)
%     - 0.032077 * np.sin(theta)
%     - 0.014615 * np.cos(2 * theta)
%     - 0.040849 * np.sin(2 * theta)
% )

% lng_deg = 132.75093  # 任意の経度(longitude)
% lng_rad = lng_deg * np.pi / 180

% lat_deg = 33.82794  # 任意の緯度(latitude)
% phi = lat_deg * np.pi / 180

% lng_diff = (lng_deg - 135) / 180 * np.pi  # 経度差


% def calc_h(dt, lng_diff, eq):
%     return (dt.hour + dt.minute / 60 - 12) / 12 * np.pi + lng_diff + eq


% # 太陽高度のarcsinの引数になる値
% def calc_sun_altitude_like(h, delta, phi):
%     return np.sin(phi) * np.sin(delta) + np.cos(phi) * np.cos(delta) * np.cos(h)


% h = calc_h(dt, lng_diff, eq)  # 時角

% sin_alpha = calc_sun_altitude_like(h, delta, phi)
% print(f"日射量:{1367*geocentri_distance_like*sin_alpha}")
% \end{lstlisting}

日射量を求めるプログラムでは, 2022年5月17日17時53分0秒における, 経度132.75093, 緯度33.82794の地点での日射量を計算している.
日射量を求めるプログラムより, 図\ref{20220523-p1}が算出される.
図\ref{20220523-p1}から, 今回与えた日時, 緯度経度における日射量は, 約299.15\si{\watt}/\si{\metre\squared}となった.

\begin{figure}
  \begin{center}
    \includegraphics[width=160mm]{sotu/figure/2/output.png}
    \caption{日射量を求めるプログラムの結果}
    \label{20220523-p1}
  \end{center}
\end{figure}

% \section{おわりに}
% 今回は, 任意の緯度経度と日時から日射量を計算するプログラムを開発した.
% 次回は実際に測定したリサイクル館の日射量計測データと, 開発したプログラムによって計算した日射量を比較することで計算精度を検証する.

% 20220529

\section{実測値と計算値のプロット}
取得したElasticsearchサーバーの日射量データと, 計算式を用いて求めた日射量データをプロットしたものを図\ref{20220529-p1}に示す.
図\ref{20220529-p1}はElasticsearchサーバーから取得した2022年6月2日の日射量データと, リサイクル館の緯度経度と日付情報(2022年6月2日)を入力として求めた日射量の計算値をプロットしている.

\begin{figure}
  \begin{center}
    \includegraphics[width=160mm]{sotu/figure/2/original-20220602-corr.png}
    \caption{2022年6月2日の日射量の実測値と計算値をプロットしたもの}
    \label{20220529-p1}
  \end{center}
\end{figure}

% 20220620

\section{相互相関を用いたずれ時間の特定}
相互相関を用いて, 実測データと計算データとの時間的遅延を検出することで, 実測データの計測日時のずれ時間を特定する.

\section{実測値データの補間}
相互相関の計算を行う前に, 実測データの計測日時の間隔を均一にするため, 線形補間を行い, 日時間隔を1 \si{\second}に統一する.

\section{相互相関による最小ラグの選定}
相互相関の計算に使用する実測データの範囲を, 2022年6月2日0時0分から2022年6月2日23時59分まで期間とする.

次に, 実測データのタイムスタンプ列を入力として求めた計算データを1分ずつスライドさせ, その都度実測値データとの相互相関を計算する.

相互相関の計算結果より, 計算値の日時を実測値より124秒進めた際に, 相互相関の値が最大となることが分かった.

しかし, 実測データには計測日時のズレは殆どないため, 0秒進めた際に相互相関の値が最大となるのが正しい.

これは, 計算データの精度が実測値と乖離していることが原因であると考えられる.

\section{日射量の計算値の精度改善}

日射量の計算値の予測精度を改善するため, 自作の日射量計算プログラムではなく, pvlibというライブラリを使用して, 任意の緯度経度と日時における日射量を求め, 相互相関を計算する.

\section{pvlibの概要}
pvlib pythonは、太陽光発電システムの性能シミュレーションや関連するタスクを実行するための関数とクラスのセットを提供する、コミュニティが開発したツールボックスである。

\section{pvlibを使って求めた計算値と実測値のプロット}
取得したElasticsearchサーバーの日射量データと, pblivを用いて求めた日射量データをプロットしたものを図\ref{2-p1}に示す.
図\ref{2-p1}はElasticsearchサーバーから取得した2022年6月2日の日射量データと, リサイクル館の緯度経度と日付情報(2022年6月2日)を入力として求めた日射量の計算値をプロットしている.

\begin{figure}
  \begin{center}
    \includegraphics[width=160mm]{sotu/figure/2/pvlib-20220602-corr.png}
    \caption{2022年6月2日の日射量の実測値と計算値をプロットしたもの}
    \label{2-p1}
  \end{center}
\end{figure}

\section{pvlibを用いて計算した日射量を使用した, 相互相関による最小ラグの選定}
相互相関の計算に使用する実測データの範囲を, 2022年6月2日0時0分から2022年6月2日23時59分まで期間とする.

次に, 実測データのタイムスタンプ列を入力として求めた計算データを1分ずつスライドさせ, その都度実測値データとの相互相関を計算する.

相互相関を求めた結果, 計算値の日時を実測値より74秒進めた際に, 相互相関の値が最大となることが分かった.

計算式を用いて求めた日射量データ用いて相互相関を計算した時と比較して, 124秒から74秒へと50秒改善している.

\section{結言}
本章では太陽光発電の計測データの補正について述べた。
次章では学内ゾーンで稼働している Elasticsearch クラスタへのデータ移行について述べる。