\chapter{付録}
%\documentclass[a4j]{jarticle} %ここは関係ない
%\usepackage{listings,jlisting} %日本語のコメントアウトをする場合jlistingが必要
%ここからソースコードの表示に関する設定
%\lstset{
%language = Python,
%basicstyle = \ttfamily\scriptsize,
%  identifierstyle={\small},
%  commentstyle={\smallitshape},
%  keywordstyle={\small\bfseries},
%  ndkeywordstyle={\small},
%  stringstyle={\small\ttfamily},
%  frame={tb},
%  breaklines=true,
%  columns=[l]{fullflexible},
%  numbers=left,
%  xrightmargin=0zw,
%  xleftmargin=0zw,
%  numberstyle={\scriptsize},
%  stepnumber=1,
%  numbersep=1zw,
%  lineskip=-0.5ex
%keywordstyle = {\bfseries \color[cmyk]{0,1,0,0}},
%}
\lstset{
	%プログラム言語(複数の言語に対応,C,C++も可)
 	language = Python,
 	%背景色と透過度
 	%backgroundcolor={\color[gray]{.90}},
 	%枠外に行った時の自動改行
 	breaklines = true,
 	%自動開業後のインデント量(デフォルトでは20[pt])	
 	breakindent = 10pt,
 	%標準の書体
 	basicstyle = \ttfamily\footnotesize,
 	%basicstyle = {\small}
 	%コメントの書体
% 	commentstyle = {\itshape \color[cmyk]{1,0.4,1,0}},
 	%関数名等の色の設定
 	classoffset = 0,
 	%キーワード(int, ifなど)の書体
% 	keywordstyle = {\bfseries \color[cmyk]{0,1,0,0}},
 	%""で囲まれたなどの"文字"の書体
% 	stringstyle = {\ttfamily \color[rgb]{0,0,1}},
 	%枠 "t"は上に線を記載, "T"は上に二重線を記載
	%他オプション:leftline,topline,bottomline,lines,single,shadowbox
% 	frame = TBrl,
	frame = topline,
 	%frameまでの間隔(行番号とプログラムの間)
 	framesep = 5pt,
 	%行番号の位置
 	numbers = left,
	%行番号の間隔
 	stepnumber = 1,
	%右マージン
 	%xrightmargin=0zw,
 	%左マージン
	%xleftmargin=3zw,
	%行番号の書体
 	numberstyle = \tiny,
	%タブの大きさ
 	tabsize = 4,
 	%キャプションの場所("tb"ならば上下両方に記載)
 	captionpos = t
}
%=========================================================================
\section{水源監視システム(送信用Pythonスクリプト)}
\centering
LoRa\_obs\_transmit.pyのソースコードを\ref{loras}に示す.
\begin{lstlisting}[label=loras,caption=LoRa\_obs\_transmit.py]
## +++***coding:utf-8***+++

import time
import os
from datetime import datetime
import serial

""" sleep()をいれて, 少し待たないとエラー落ちする """
time.sleep(60)

class Main() :
    
    def __init__(self):
        """ 初期値および対象ディレクトリの設定 """
        self.s_num = 0

        self.copy_dir = "C:/Users/taikimizukan/Dropbox/sumitomo/obscsv/" 
        self.target_dir = "C:/Users/taikimizukan/Desktop/obs_data/"              
        self.temporary_log = "./temporary_log.txt"

        """ 最終データを取得 """
        with open(self.temporary_log,"r") as f :
            self.old_line = f.readline()
            print("前回のデータ:"+str(self.old_line))
               
        """ 起動時 [$RFINF,ON***]コマンド送信 """
        INF = "$RFINF,ON***"
        with serial.Serial("COM7",115200,timeout=2) as ser :
            time.sleep(2)
            while True :
                for i in INF :
                    ser.write(i.encode("utf-8"))

                result = str(ser.readline())
                if result.find("RESULT,RFINF,ON,OK") > 0 :
                    break
                else :
                    time.sleep(2)

        """ ループ関数実行 """
        self.Loop()
        

    """ 作成日時が最新ファイルのフルパスを取得し返す関数 """
    def get_file_path(self, target_dir) : 
        """ 対象ディレクトリ下の.datファイルのパスを取得し, [target_files]に納める """
        target_files = []
        for root, dir, files in os.walk(target_dir) :
            target_file = [os.path.join(root,f) for f in files if f.endswith(".dat")]# .txt -> .datへ
            target_files.extend(target_file)
        """ 取得した.datファイルのフルパスに作成時間を足してリストに納める """
        file_ctime = []
        for f in target_files :
            file_ctime.append((f,os.path.getctime(f)))        
        """ 取得時間でソートし最新の.datファイルのパスのみ返す"""
        sorted_file_ctime = sorted(file_ctime,key=lambda x :x[1])
                           
        return sorted_file_ctime[len(sorted_file_ctime)-1][0]
        
    """ 最終行を取得, シークエンス番号を加えてコピー """    
    def check_copy(self):
##        name = self.target_file.replace(self.target_dir,"")
        with open(self.target_file,"r") as f :
            """ ファイルデータを全て読み込, 最終行だけを取得 """
            lines = f.readlines()
            if len(lines) > 0 :
                line = lines[len(lines)-1]
            else :
                line = self.old_line
            
        """ 最終行が前回のものと異なるか? """
        if line != self.old_line :
            
            """ シークエンス番号を追加 """
            self.s_num += 1
            
            file_name = line.split(",")
            file_name = "obs_" +"".join(file_name[0:3])
            self.ymd = "".join(file_name[0:3])

            self.old_line = line

            """ Dropboxにコピー """
            with open(self.copy_dir+file_name+".csv","a") as cf :
                data = line.strip() + "," + str(self.s_num)+"\n"
                cf.write(str(data))
            
            """ 最終行を保存 """
            with open(self.temporary_log,"w") as f :
                f.write(line)

            self.arduino_serial(data)
            time.sleep(5)
            self.TxMSG()
            self.Ping()
                    
        else :
            print("Not updated")
            pass

    """ arduinoを経由してLoRaにデータを送信する関数 """
    def arduino_serial(self,d) :
        print("---"*5 + "arduino_serial" + "---"*5)
        buf = 0
        with serial.Serial("COM7",115200,timeout=1) as ser :
            """ ポートを開いて少し待機が必要 """
            time.sleep(2)            
            """ごみの吸出し"""
            buf = ser.readlines()                       
            d = d.strip()
            """ 送信コマンドの形に """
            d = "$RFSND,0004,"+d+"***"
            print("To arduino Data --> " +d)
            
            """ Python(PC) -> arduino -> LoRa だと1文字ずつ送らないといけない? """
            for i in d :
                ser.write(i.encode("utf-8"))                

        """ 0009 : 第二中継機にダミーMSGをとばす, 戻り値を保存 """
    def TxMSG(self) :
        target_add = "0009"
        self.now = datetime.now().strftime("%Y,%m,%d,%H,%M,%S")
        self.today = datetime.today().strftime("%Y%m%d")
        
        msg = "$RFSND,{0},{1},{2},{2}***".format(target_add,self.now,self.counter)

        with serial.Serial("COM7",115200,timeout=15) as ser :
                time.sleep(2)
                for i in msg :
                        ser.write(i.encode("utf-8"))
                        time.sleep(0.05)
                print(ser.readline().decode("utf-8"))
                res = ser.readline().decode("utf-8")

        if len(res) > 10 :
                res = res.replace(" ",",").replace("*",",").replace(":",",")
                with open("C:/Users/taikimizukan/Dropbox/sumitomo/RSSI_CHECK_TX/rssi_tx_obs_{}.csv".format(str(self.today)),"a") as f :
                    f.write(res+"\n")                  
        else :
                pass

    """発電所のLoRaにpingを送って生存確認"""
    def Ping(self) :
        PING = "$RPING,0004***"
        with serial.Serial("COM7",115200,timeout=10) as ser :
            time.sleep(2)
            for i in PING :
                ser.write(i.encode("utf-8"))
                time.sleep(0.05)

            print(ser.readline().decode("utf-8"))
            res_ping = ser.readline().decode("utf-8")
                
        if len(res_ping) > 10 :
            print(res_ping)
            now = datetime.now().strftime("%Y,%m,%d,%H,%M,%S")
            with open("C:/Users/taikimizukan/Dropbox/sumitomo/PING/ping_{}.csv".format(str(self.today)),"a") as f :
                    f.write(str(now)+","+str(self.s_num)+"," + str(res_ping))
        else :
            pass

    """ 繰り返し """
    def Loop(self):  
        while True:
            try :
                time.sleep(20)
                self.target_file = self.get_file_path(self.target_dir)

                self.check_copy()
            
            except Exception as E :
                with open("./Error_Log.txt","a") as ef :
                    ef.write(str(E))


if __name__ == "__main__" :
    Main()
\end{lstlisting}
%\end{document}
%=========================================================================
\clearpage
\section{水源監視システム(受信用Pythonスクリプト)}
\centering
LoRa\_obs\_receive.pyのソースコードを\ref{lorar}に示す.
\begin{lstlisting}[label=lorar,caption=LoRa\_obs\_raceive.py]
## coding:utf-8

import paho.mqtt.client as mqtt
import serial
from datetime import datetime
import time

class Main() :
    def __init__(self) :
        self.INF_Input()
        self.Loop()

    def INF_Input(self) :
    """ 起動時 [$RFINF,ON***]コマンド送信 """
    
        INF = "$RFINF,ON***"
        with serial.Serial("COM3",115200,timeout=2) as ser :
            time.sleep(2)
            while True :
                for i in INF :
                    ser.write(i.encode("utf-8"))

                result = str(ser.readline())
                if result.find("RESULT,RFINF,ON,OK") > 0 :
                    print("RFINF,OK")
                    break
                else :
                    time.sleep(2)

    def LoRa_Receive(self) :
        try :
        """ LoRaからデータを読み込む """
            while True :
                with serial.Serial("COM3",115200,timeout=120) as ser :
                    res = ser.readline().decode("utf-8")
                    if len(res) > 15 and res.find("RFRX") > 0 and res.find("RECEIVED") < 0 :
                        print("==="*25)
                        print("RXData  ==> "+res)
                        break
                    
                    else :
                        print("==="*25)
                        print("else_data=>"+res)
                        
            """ 必要なデータを取り出す """
            res_list = res.split("*")[0].split(",")[1:]
            data = ",".join(res_list)
            add = res_list[0]

            """ データの日付を確認(ymd) """
            ymd = "".join(res_list[1:4])


            return = res, data, add, ymd

        except Exception as E :
            now = datetime.now().strftime("%y/%m/%d %H:%M:%S")
            with open("LoRa_Receive_Error_LOG.txt","a") as ef :
                    ef.write(now +" : "+ str(E)+"\n")
            self.Loop()

                        

    def MQTT_Publish(self, res):
    """ MQTT情報(publish) """
        host = "133.71.***.***"
        port = 1883
        topic = "***********"
        """ MQTT-Publish """
        try :   
            print("publish ==> " +str(res))
            client = mqtt.Client(protocol=mqtt.MQTTv311)
            client.connect(host,port=port,keepalive=10)
            client.publish(topic,res)
                        
        except Exception as E :
            now = datetime.now().strftime("%y/%m/%d %H:%M:%S")
            with open("Publish_Error_LOG.txt","a") as ef :
                    ef.write(now +" : "+ str(E)+"\n")
                
    def Storage(self,res,data,ymd):
        try :
            """ メタデータを保存 """
            with open("./meta_data/meta_data_{}.txt".format(ymd),"a") as f :
                f.write(res+"\n")

            """ データを保存 """
            with open("./data/data_{}.txt".format(ymd),"a") as f :
                f.write(data+"\n")

            """ Dropboxに保存 """
            with open("C:/Users/sumitomo02/Dropbox/test_folder/RX/DATA/RX_{}.txt".format(ymd),"a") as f :
                f.write(data+"\n")

            with open("C:/Users/sumitomo02/Dropbox/test_folder/RX/META_DATA/RX_{}.txt".format(ymd),"a") as f :
                f.write(res+"\n")
                
        except Exception as E :
            now = datetime.now().strftime("%y/%m/%d %H:%M:%S")
            with open("Storage_Error_LOG.txt","a") as ef :
                ef.write(now +" : "+ str(E)+"\n")
            print(E)
            
    def Loop(self) :
        while True :
            res, data, add, ymd = self.LoRa_Receive()
            self.Storage(res, data, ymd)
            self.MQTT_Publish(res)
            self.Ping(add)
            

main = Main()
\end{lstlisting}